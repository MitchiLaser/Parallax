\section{Operational Amplifier}

Operational Amplifiers, or op-amps, are highly used active electronic component\footnote{Active components need a supply voltage whereas passive components don't need a connection to a power supply} due to their versatility. In this section only the topics relevant for the application of this project are discussed. Let's take a look at the operational amplifier in \cref{fig:op_amp}: It has two inputs, namely an inverting and a non-inverting one, along with an output.

\begin{figure}[htb]
	\centering
	\begin{tikzpicture}
		\draw (0,0) node[op amp] (opamp) {}
			(opamp.+) -- + (-0.5,0) node[left] {$U_+$}
			(opamp.-) -- + (-0.5,0) node[left] {$U_-$}
			(opamp.out) -- + (0.5,0) node[right] {$U_{out}$}
			(opamp.up) -- + (0,0.5) node[vcc] {$V_{cc}$}
			(opamp.down) -- + (0,-0.5) node[vee] {$V_{ee}$}
		;
	\end{tikzpicture}
	\caption{Operational Amplifier}
	\label{fig:op_amp}
\end{figure}

Op-amps are being used in a configuration where the output voltage is connected to an input which creates a feedback loop. The behavior of am ideal operational amplifier can be described by the \glqq{}golden Rules\grqq{}:
\begin{enumerate}
	\item No current flows into the inputs.
	\item The output strives to minimize the voltage difference between the two input pins to zero volts.
\end{enumerate}

These rules only apply within a feedback loop. It's important to notice that the first rule is just an approximation because the input resistance cannot be infinitely high, resulting in an input bias current in reality. We assume that this is true in our model to describe the functionality of an op-amp and for practical applications the input impedance can be looked up in the data sheet. There are two op-amp circuits being used for the oscilloscope: the voltage follower and the non-inverting amplifier.

\subsection{Op-Amp as a Voltage follower}

The voltage follower configuration, illustrated in \cref{fig:voltage-follower}, serves as an impedance converter or as a buffer.

\begin{figure}[htb]
	\centering
	\begin{tikzpicture}
		\draw (0,0) node[op amp] (opamp) {}
			(opamp.out) -- ++ (0.5,0) node[right] {$U_{out}$}
			(opamp.up) -- ++ (0,0.5) node[vcc] {$V_{cc}$}
			(opamp.down) -- ++ (0,-0.5) node[vee] {$V_{ee}$}
			(opamp.-) -- ++ (-0.5,0) node[left] {$U_-$}
			-- ++ (0,2)
			-| (opamp.out) node {$\bullet$}
			(opamp.+) ++ (-0.5,-1.5) node[ground] (gnd) {}
			(opamp.+) -- ++ (-0.5,0) node[left] {$U_+$}
			-- ++ (0,-0.5)
			to[sV, l_=$U_{in}$] (gnd)
		;
	\end{tikzpicture}
	\caption{Op-Amp as a Voltage follower}
	\label{fig:voltage-follower}
\end{figure}

In this setup, the output voltage mirrors the input voltage, serving as a buffer. The high input impedance of the op-amp enables voltage replication without drawing significant current from the source, while the low output impedance ensures efficient transmission of the replicated voltage.

An example illustrates the significance of this buffer: a voltage divider generates a reference voltage, but using it as a direct voltage source is impractical due to the voltage drop when attaching a load. Using a voltage follower, as shown in Figure \ref{fig:attenuator-buffer}, solves this issue and allows the voltage follower to supply current while maintaining an output voltage equivalent to the voltage divider's reference voltage. In this sketch the power supply connections of the op-amp have been left out. This is done a lot of times in circuits because the connection to a power supply is necessary for an op-amp and therefore has not to be specialised separately.

\begin{figure}[htb]
	\centering
	\begin{tikzpicture}
		\draw (0,0) node[op amp] (opamp) {}
			(opamp.-) -- ++ (0,1)
			-| (opamp.out) node {$\bullet$}
			-- ++ (0.5,0) node[right] {$U_{out}$}
			(opamp.+) -- ++ (-1,0) node {$\bullet$} coordinate (A)
			to[resistor, l=$R_1$] ++ (0,2) node[vcc] {$V_{cc}$}
			(A) ++ (0,-2) node[ground] {} to[resistor, l=$R_2$] (A)
		;
	\end{tikzpicture}
	\caption{Voltage follower as a buffer}
	\label{fig:attenuator-buffer}
\end{figure}

When the input voltage $V_{cc}$ is constant and has no time dependency it is recommended to add a capacitor at the non-inverting input of the op-amp. This would stabilize the voltage and filter out minor fluctuations and instabilities.

\subsection{Op-Amp as a non-inverting amplifier}

Amplification of input signals is another essential task for an op-amp. There are multiple ways to perform an amplification. Here the non-inverting amplifier circuit, represended in \cref{fig:non_inverting_amplifier} is going to be explained.

\begin{figure}[htb]
	\centering
	\begin{tikzpicture}
		\draw (0,0) node[op amp, noinv input up] (opamp) {}
			(opamp.out) -- ++ (1,0) node[right] {$U_{out}$}
			(opamp.out) ++ (0.5,0) node {$\bullet$}
			to[resistor, l=$R_1$] ++ (0,-2) node {$\bullet$} coordinate (A)
			to[resistor, l=$R_2$] ++ (0,-2) node[ground] (gnd) {}
			(A) -| (opamp.-)
			(opamp.+) -- ++ (-1,0)
			++ (0,-2) node[ground] (gnd) {}
			to[sV, l=$U_{in}$] ++ (0,2)
		;
	\end{tikzpicture}
	\caption{Op-Amp as a non-inverting amplifier}
	\label{fig:non_inverting_amplifier}
\end{figure}

How does this circuit work? Based on the rules of the ideal op-amp the inverting input voltage $U_-$ is going to match the voltage at the non-inverting input $U_+$. $U_+$ is determined by the voltage across the two resistors of the attenuator. Therefore we can set $U_0$ from \cref{fig:simple-voltage-divider} equal to the input voltage:
\begin{equation}
	U_{in} = U_{out} \cdot \frac{R_2}{R_1 + R_2}\,.
\end{equation}
This equation can be transformed to calculate $U_{out}$ based on an arbitrarily chosen $U_{in}$:
\begin{equation}
	U_{out} = U_{in} \cdot \frac{R_1 + R_2}{R_2} = U_{in} \cdot \left( 1 + \frac{R_1}{R_2} \right)\,.
\end{equation}
The amplification factor, formally known as the \glqq{}gain\grqq{}, cannot be lower than $\SI{1}{}$. For the oscilloscope this configuration can be used to amplify small signals and map them to a wider voltage range, allowing to use the full input range of an ADC for their measurement.

While the \glqq{}inverting amplifier\grqq{} circuit is more common in electronic applications, the non-inverting configuration is preferable in this project for several reasons:
\begin{itemize}
	\item The non-inverting amplifier has a higher input impedance, minimizing voltage drops when measuring analog voltages.
	\item It eliminates the need for a negative voltage source $V_{ee}$, which simplifies the circuit.
	\item The non-inverting amplifier maintains isolation between the input and output signals. This benefits makes it simple to get the overall input impedance for the signal inputs of the oscilloscope.
\end{itemize}
