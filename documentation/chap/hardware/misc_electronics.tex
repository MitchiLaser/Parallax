\section{Further electrical components and their usage}

\subsection{Voltage clamping with Diodes}

A Diode is a semiconductor device that allows current to flow in one direction only\footnote{This is technically not true because a diode can also let current flow in the opposite direction but this aspect is not necessary for the design described in this report}. When a Voltage is applied across the two terminals in the forward direction the current flows easily while in reverse direction the diode blocks current. In the forward direction current flows only if the applied voltage exceeds a threshold limit, the forward voltage, typically around $\SI{0.6}{V}$ to $\SI{0.7}{V}$ for standard silicon diodes. This also means that in forward direction a voltage has a constant voltage drop which is equal to its forward voltage. Diodes are commonly used to rectify electrical voltages with various types like Light-emitting Diodes (LEDs), Zener diodes and Schottky diodes. \cref{fig:diode-types} shows symbols for some diode types.

\begin{figure}[htb]
	\centering
	\begin{tikzpicture}
		\draw (0,0) to[Do] ++ (2,0)
			(4,0) to[zDo] ++ (2,0)
			(8,0) to[sDo] ++ (2,0)
		;
	\end{tikzpicture}
	\caption{Diode types (left to right): standard diode, Z-Diode, Schottky-Diode}
	\label{fig:diode-types}
\end{figure}

We want to use a diode to prevent our circuit from input voltages which exceed the designed range. \cref{fig:clamping-diodes} illustrates this: When the signal line is below $\SI{0}{V}$ current flows directly to ground from $D_2$. Similar, when the signal line exceeds the supply voltage $V_{cc}$, $D_1$ conducts. However this only works when the input impedance of the device $U_{out}$ is connected to is lower than the resistance across the diode. This clamping actions maintains the voltage within a defined range.

\begin{figure}[htb]
	\centering
	\begin{tikzpicture}
		\draw (0,0) node[ground] {} 
			to[sV, l=$U_{in}$] ++ (0,2)
			-- ++ (0.5,0)
			to[R, l=R] ++ (2,0)
			-- ++ (1,0) node {$\bullet$} coordinate (A)
			to[short, -o] ++ (1,0) node[right] {$U_{out}$}
			(A) to[Do, l=$D_1$] ++ (0,2)
			node[vcc] {$V_{cc}$}
			(A) ++ (0,-2) node[ground] {}
			to[Do, l=$D_2$] (A)
		;
	\end{tikzpicture}
	\caption{A visualisation of the usage of clamping diodes}
	\label{fig:clamping-diodes}
\end{figure}

There is one disadvantage when this circuit is build with normal diodes: The voltage between the two terminals must be greater than the forward voltage. In this Example the voltage can only be clamped between $\SI{-0,7}{V}$ and $V_{cc} + \SI{0,7}{V}$. Diodes with a smaller forward voltage reduce this gap and here Schottky diodes are the better choice. Normal diodes are build by combining a p-type and a n-type doped semiconductor. Schottky diodes are just a combination of a doped semiconductor and a raw piece of metal. This construction has faster switching times and lower forward voltages, down to $\SI{0,3}{V}$.

When selecting diodes, their current tolerance is critical. The resistor $R$ in Figure \ref{fig:clamping-diodes} limits current flow to prevent diode damage from excessive heat dissipation. Besides clamping, these diodes also function as Electrostatic Discharge (ESD) protection. ESD, a sudden flow of electricity between charged objects\footnote{ESD events are the causes for electrical shocks when touching a metal object after walking over a carpet with socks.}, poses a risk to electronic components because the discharge voltage between an ESD event easily reaches values of kolivolts whereas electronic components only operate in ranges of a few volts. Diodes prevent damage by clamping voltages within safe ranges during ESD events.

\subsection{Analog Multiplexers}

An analog multiplexer is an electronic component acting like a switch, routing multiple signals and a common input or output. Only one of the terminals can be connected at a time. Selection occurs via input signal lines which determine the connected terminal by the binary representation of its terminal number.

On the oscilloscope a multiplexer switches between resistors which are connected to ground. This makes the gain of a non-inverting amplifier variable across pre-defined configurations.

\subsection{Low pass filters}

A low pass filter is a combination of a resistor and a capacitor as depicted in \cref{fig_lowpass}. When a high frequency signals comes into this circuit form the left side then the capacitor is charged. This results in a smothering of the input signal. A low frequency signal gets less distorted than a high frequency signal.

\begin{figure}[htb]
	\centering
	\begin{tikzpicture}
		\draw (0,0) -- ++ (0.5,0)
			to[resistor, l=$R$] ++ (2,0)
			-- ++ (0.5,0)
			to[capacitor, l=$C$, *-] ++ (0,-1.5)
			node[ground] {}
			++ (0,1.5) -- ++ (1,0)
		;
	\end{tikzpicture}
	\caption{Low-Pass filter}
	\label{fig_lowpass}
\end{figure}

This circuit can be used to filter out some high frequency parts of a signal above the cutoff frequency:
\[
f = \frac{1}{2 \cdot \pi \cdot R \cdot C}\,.
\]
A low pass filter is going to be used to remove unwanted higher order frequency components of the input signal and removes aliasing effects. The reason for them is the Nyquist-Shannon theorem which in simple terms says that the highest with an oscilloscope measurable frequency is half of the sampling rate.
