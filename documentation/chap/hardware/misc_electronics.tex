\section{Further electrical components and their usage}

\subsection{Voltage clamping with Diodes}

A Diode is a semiconductor device that allows current to flow in one direction only\footnote{This is technically not true because a diode can also let current flow in the opposite direction but this aspect is not necessary for the design described in this report}. When a Voltage is applied across the two terminals in the forward direction the current flows easily while in reverse direction the diode blocks current. In the forward direction current flows only if the applied voltage exceeds a threshold limit, the forward voltage, typically around $\SI{0.6}{V}$ to $\SI{0.7}{V}$ for standard silicon diodes. This also means that in forward direction a voltage has a constant voltage drop which is equal to its forward voltage.

Regularly a diode can be used to prevent large voltage spikes from damaging sensitive electronic components, either by clamping the voltage to a safe level or by redirecting the current away from sensitive parts of the circuit. This is often done in power supply circuits to protect against overvoltage conditions. \cref{fig:clamping-diodes} illustrates this: When the signal line is below $\SI{0}{V}$ current flows directly to ground from $D_2$. Similar, when the signal line exceeds the supply voltage $V_{cc}$, $D_1$ conducts. However this only works when the input impedance of the device $U_{out}$ is connected to is lower than the resistance across the diode. This clamping actions maintains the voltage within a defined range.

\begin{figure}[htb]
	\centering
	\begin{tikzpicture}
		\draw (0,0) node[ground] {} 
			to[sV, l=$U_{in}$] ++ (0,2)
			-- ++ (0.5,0)
			-- ++ (2,0) node {$\bullet$} coordinate (A)
			to[short, -o] ++ (1,0) node[right] {$U_{out}$}
			(A) to[Do, l=$D_1$] ++ (0,2)
			node[vcc] {$V_{cc}$}
			(A) ++ (0,-2) node[ground] {}
			to[Do, l=$D_2$] (A)
		;
	\end{tikzpicture}
	\caption{A visualisation of the usage of clamping diodes}
	\label{fig:clamping-diodes}
\end{figure}

There is one disadvantage when this circuit is build with normal diodes: The voltage between the two terminals must be greater than the forward voltage. In this Example the voltage can only be clamped between $\SI{-0,7}{V}$ and $V_{cc} + \SI{0,7}{V}$. Diodes with a smaller forward voltage reduce this gap and here Schottky diodes are the better choice. Normal diodes are build by combining a p-type and a n-type doped semiconductor. Schottky diodes are just a combination of a doped semiconductor and a raw piece of metal. This construction has faster switching times and lower forward voltages, down to $\SI{0,3}{V}$.

After some more thinking the clamping didoes were removed from the design for the following reasons:
\begin{itemize}
	\item The clamping didoes chosen for this design were Schottky diodes. Unfortunately these have a relatively high leakage current which messes up the measurement for the smallest input range.
	\item The input current maximum rating of the ADC is only exceeded for input voltages larger than \SI{300}{V}, which should never happen under normal operating conditions. For ESD protection, the Op-Amp's internal clamping diodes are sufficient.
	\item The ADC's input cannot exceed the supply voltage $V_{cc}$ and $GND$ of the Op-Amp.
\end{itemize}

\subsection{Analog Multiplexers}

An analog multiplexer is an electronic component acting like a switch, routing multiple signals and a common input or output. Only one of the terminals can be connected at a time. Selection occurs via input signal lines which determine the connected terminal by the binary representation of its terminal number.

On the oscilloscope a multiplexer switches between resistors which are connected to ground. This makes the gain of a non-inverting amplifier variable across pre-defined configurations.

\subsection{Low pass filters}

A low pass filter is a combination of a resistor and a capacitor as depicted in \cref{fig_lowpass}. When a high frequency signals comes into this circuit form the left side then the capacitor is charged. This results in a smothering of the input signal. A low frequency signal gets less distorted than a high frequency signal.

\begin{figure}[htb]
	\centering
	\begin{tikzpicture}
		\draw (0,0) -- ++ (0.5,0)
			to[resistor, l=$R$] ++ (2,0)
			-- ++ (0.5,0)
			to[capacitor, l=$C$, *-] ++ (0,-1.5)
			node[ground] {}
			++ (0,1.5) -- ++ (1,0)
		;
	\end{tikzpicture}
	\caption{Low-Pass filter}
	\label{fig_lowpass}
\end{figure}

This circuit can be used to filter out some high frequency parts of a signal above the cutoff frequency:
\[
f = \frac{1}{2 \cdot \pi \cdot R \cdot C}\,.
\]
A low pass filter is going to be used to remove unwanted higher order frequency components of the input signal and removes aliasing effects. The reason for them is the Nyquist-Shannon theorem which in simple terms says that the highest with an oscilloscope measurable frequency is half of the sampling rate.
