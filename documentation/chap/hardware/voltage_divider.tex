\section{Voltage Divider}

One of them most important circuits in electronics is the voltage divider as depicted in \cref{fig:simple-voltage-divider}. In this circuit two resistors $R_1$ and $R_2$ are between connected in series between a voltage source $V_{cc}$ and Ground. Because this is a closed circuit there is a current flowing through the resistors and each resistor is responsible for an individual voltage drop. Can we calculate the voltage $U_o$ between these two resistors?

It is simple to see that the current flowing in this circuit is the same for both resistors. When two resistors are connected in series their resistances sum up and this total resistance can be used to calculate the current with Ohms law:
\begin{equation*}
	I = \frac{U}{R} = \frac{V_{cc}}{R_1 + R_2}\,.
\end{equation*}
We also know that the voltage drop $U_1$ across $R_1$ and the voltage drop $U_2$ across $R_2$ sum up to the supply voltage: $U_1 + U_2 = V_{cc}\,$. The voltage drop across $R_2$ can be calculated using Ohms law again because we know the current and the resistance:
\begin{equation*}
	U_2 = R \cdot I = V_{cc} \cdot \frac{R_2}{R_1 + R_2}\,.
\end{equation*}
A similar behavior can be seen for $R_1$:
\begin{equation*}
	U_1 = R \cdot I = V_{cc} \cdot \frac{R_1}{R_1 + R_2}\,.
\end{equation*}

This shows us that we can calculate the voltage $U_o = U_2 = V_{cc} - U_1$ between the resistors. It is independent from the total current and only depending on the ratio of the two resistors. It would not make a difference if we use $R_1 = \SI{10}{K\Omega}, R_2 = \SI{2,2}{K\Omega}$ or $R_1 = \SI{10}{\Omega}, R_2 = \SI{2,2}{\Omega}$, the result would be the same.


\begin{figure}[htb]
	\centering
	\begin{tikzpicture}
		\draw (0,0) node[ground] {}
			to[resistor, l=$R_2$, -*] ++ (0,2) coordinate (A)
			to[short, -o] ++ (1,0)
			node[auto, anchor=west] {$U_o$}
			(A) to[resistor, l=$R_1$] ++ (0,2)
			node[vcc] {$V_{cc}$}
		;
	\end{tikzpicture}
	\caption{A voltage divider made with two resistors}
	\label{fig:simple-voltage-divider}
\end{figure}


The characteristic equation to describe a voltage divider can be seen in \cref{equ:resistive-voltage-divider}:
\begin{equation}
	\frac{U_1}{U_2} = \frac{R_1}{R_2}\,.
	\label{equ:resistive-voltage-divider}
\end{equation}

We can use this circuit to reduce a voltage but its usability as a voltage source is limited. This can be shown in \cref{fig:loaded-voltage-divider} by adding a load, in this case a resistor $R_L$. Now the total resistance of the parallel resistors $R_2$ and $R_L$ calculates to
\begin{equation*}
	R_x = \frac{1}{\frac{1}{R_2} + \frac{1}{R_L}}
\end{equation*}
and the result of this is smaller than $R_2$ and $R_L$. The load changes the voltage divider and this changes the output voltage $U_o$.


\begin{figure}[htb]
	\centering
	\begin{tikzpicture}
		\draw (0,0) node[ground] {}
			to[resistor, l=$R_2$, -*] ++ (0,2) coordinate (A)
			to[short, -o] ++ (1,0) coordinate (B)
			node[auto, anchor=west] {$U_o$}
			(A) to[resistor, l=$R_1$] ++ (0,2)
			node[vcc] {$V_{cc}$}
			(B) to[resistor, l=$R_L$] ++ (0,-2)
			to[short, -*] ++ (-1,0)
		;
	\end{tikzpicture}
	\caption{A voltage divider with load resistance}
	\label{fig:loaded-voltage-divider}
\end{figure}

A voltage divider can only be used as a voltage source if there is no load attached to it. This means we cannot use it to drive an electric device but we will later see that there is a possibility to avoid this issue.

Right now we only have discussed what happens when we have a fixed voltage $V_{cc}$ driving this circuit. With an AC supply $U = U_0 \cdot \sin(\omega \cdot t)$ the voltage divider still works the same because only $V_{cc}$ is time dependant, the resistors are constant and independent from the voltage or the frequency of $V_{cc}$ and therefore it should work the same with AC and DC current.

Hold on! The last paragraph was partially a lie. In theory a resistor should only have a resistance but in reality every component has some resistance, capacitance and inductance. In many cases we can ignore the unwanted side-effects but when the circuit has to interact with high frequency signals these have to be taken into account. If we want to measure the voltage $U_o$ then our measurement device (which would be an ADC) has a large input resistance and a small input capacity. The large input resistance is good because ifs effect on the voltage divider can be neglected\footnote{You wanna know why? Calculate it with a 1:1 voltage divider made out of $\SI{1}{K\Omega}$ resistors and load resistor $R_L = \SI{10}{M\Omega}$.} but the small input capacitance would introduce a new side effect on this circuit. \cref{fig:voltage-divider-distortion} is a simulation of a square signal which will be distorted by the small input capacitance ($C_1 = \SI{15}{pF}$) of a measurement device (The large input resistance $R_3 = \SI{1}{M\Omega}$ was also taken into account in this simulation). Before we discuss how to deal with this side effect we will take a look on a voltage divider which is build especially for AC currents.

\begin{figure}[htb]
		\centering
		\includegraphics[width=0.8\textwidth]{voltage-divider-distortion-simulation.png}
		\caption{A square wave signal which is being disturbed by the input capacitance of an ADC}
		\label{fig:voltage-divider-distortion}
\end{figure}


\subsection{The capacitive voltage divider}

For a DC voltage a capacitor is an impervious barrier but for an AC voltage the capacitor behaves like an electrical conductor which induces a phase shift of $\SI{90}{\degree}$. If we arrange two capacitors as seen in \cref{fig:capacitive-voltage-divider} and have an alternating voltage $\sim V_{cc}$ we can observe an similar behavior as with the resistor based voltage divider in \cref{fig:simple-voltage-divider}.


\begin{figure}[htb]
	\centering
	\begin{tikzpicture}
		\draw (0,0) node[ground] {}
			to[C, l=$C_2$, -*] ++ (0,2) coordinate (A)
			to[short, -o] ++ (1,0)
			node[auto, anchor=west] {$U_o$}
			(A) to[C, l=$C_1$] ++ (0,2)
			node[vcc] {$\sim V_{cc}$}
		;
	\end{tikzpicture}
	\caption{A capacitive voltage divider}
	\label{fig:capacitive-voltage-divider}
\end{figure}


In this circuit the overall charge $Q$ through the capacitors is preserved:
\begin{equation*}
	Q = C \cdot U = C_1 \cdot U_1 = C_2 \cdot U_2\,
\end{equation*}
where $U_1$ and $U_2$ are the voltage drops across the capacitors $U_1$ and $U_2$. With the overall capacity of two capacitors in series:
\begin{equation*}
	C_x = \frac{1}{\frac{1}{C_1} + \frac{1}{C_2}} = \frac{C_1 \cdot C_2}{C_1 + C_2}
\end{equation*}
and the use of $Q = C_x \cdot V_{cc} = C_2 \cdot U_2$ the voltage $U_2$ drop across $C_2$ can be calculated:
\begin{equation*}
	U_2 = U_0 \frac{Q}{C} =  \frac{V_{cc}}{C_2} \cdot C_x = \frac{V_{cc}}{C_2} \cdot \frac{C_1 \cdot C_2}{C_1 + C_2} = V_{cc} \cdot \frac{C_1}{C_1 + C_2}
\end{equation*}
From symmetric reasons the equation for $U_1$ looks similar:
\begin{equation*}
	U_1 =  V_{cc} \cdot \frac{C_1}{C_1 + C_2}
\end{equation*}

From this the characteristic equation of the capacitive voltage divider is \cref{equ:capacitive-voltage-divider}

\begin{equation}
	\frac{U_1}{U_2} = \frac{C_2}{C_1}\,.
	\label{equ:capacitive-voltage-divider}
\end{equation}

With this knowledge we can build a voltage divider for AC signals but it cannot be used for DC voltages. Like a resistive voltage divider a capacitive one also only works when there is no load attached. A load can also be seen as an additional capacitor, similar to \cref{fig:loaded-voltage-divider}, which changes the capacitance ration and affects the voltage drop.


\subsection{Frequency compensated attenuator}

A resistive attenuator can only be used properly with DC voltages or AC voltages at low frequencies and a capacitive attenuator can be used with alternating voltages at high frequencies but is unusable for DC voltages. The solution to achieve both within a single circuit is called a "Frequency compensated attenuator" (\textit{attenuator} is just another name for a voltage divider).


\begin{figure}[htb]
	\centering
	\begin{tikzpicture}
		\draw (0,0) node[ground] {}
			to[short, *-] ++ (-2,0)
			to[resistor, l=$R_2$] ++ (0,2) coordinate (A)
			to[resistor, l=$R_1$, *-] ++ (0,2)
			-- ++ (2,0)
			node[vcc] {$V_{cc}$}
			to[short, *-] ++ (0,0)
			to[C, l=$C_1$, -*] ++ (0,-2) coordinate (B)
			to[C, l=$C_2$, -*] ++ (0,-2)
			-- ++ (-1,0)
			(B) ++ (2,0) coordinate (C)
			to[vC, l=$C_T$, *-] ++ (0,-2)
			to[short, -*] ++ (-2,0)
			(A) to[short] (B)
			to[short] (C)
			to[short, -o] ++ (1,0)
			node[auto, anchor=west] {$U_o$}
		;
	\end{tikzpicture}
	\caption{A frequency compensated attenuator with an adjustable capacitor for fine tuning}
	\label{fig:frequency-compensated-attenuator}
\end{figure}

The attenuator in \cref{fig:frequency-compensated-attenuator} is just a combination of a resistive voltage divider for DC signals and a capacitive voltage divider for AC signals in parallel. By using resistors which are way smaller than the load resistance of a measurement device the load would not affect this attenuator. The same thing happens when the capacitors are chosen large enough: a small load capacity would not effect the attenuator. There is also a trimmer capacitor which will be discussed later.

Let's take a look at the behaviour of this voltage divider first. We know the descriptive formulas of the previously discussed voltage dividers:
\begin{equation*}
	\frac{U_1}{U_2} = \frac{R_1}{R_2} = \frac{C_2}{C_1}
\end{equation*}
This means that a frequency compensated attenuator needs to fulfill the following requirements:
\begin{equation}
	R_1 \cdot C_1 = R_2 \cdot C_2\,.
\end{equation}
This equation is essential for choosing the right components. One missing detail lies in the trimmer capacitor. Its only task is to compensate computational mishaps or other side effects which have not been taken into account. This trimmer capacitor can be set to its proper value by a calibration process. To do so the attenuator needs to be driven by a square signal. During a recording of $U_o$ the trimmer capacitor needs to be adjusted until the results show a pure square signal with no distortions.
