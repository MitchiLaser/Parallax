\section{Voltage Divider}

One of them most important circuits in electronics is the voltage divider as depicted in \cref{fig:simple-voltage-divider}. In this circuit two resistors $R_1$ and $R_2$ are between connected in series between a voltage source $V_{cc}$ and Ground. Because this is a closed circuit there is a current flowing through the resistors and each resistor is responsible for an individual voltage drop. Can we calculate the voltage $U_o$ between these two resistors?

It is simple to see that the current flowing in this circuit is the same for both resistors. When two resistors are connected in series their resistances sum up and this total resistance can be used to calculate the current with Ohms law:
\begin{equation*}
	I = \frac{U}{R} = \frac{V_{cc}}{R_1 + R_2}\,.
\end{equation*}
We also know that the voltage drop $U_1$ across $R_1$ and the voltage drop $U_2$ across $R_2$ sum up to the supply voltage: $U_1 + U_2 = V_{cc}\,$. The voltage drop across $R_2$ can be calculated using Ohms law again because we know the current and the resistance:
\begin{equation}
	U_2 = R \cdot I = V_{cc} \cdot \frac{R_2}{R_1 + R_2}\,.
\end{equation}
A similar behavior can be seen for $R_1$:
\begin{equation}
	U_1 = R \cdot I = V_{cc} \cdot \frac{R_1}{R_1 + R_2}\,.
\end{equation}

This shows us that we can calculate the voltage $U_o = U_2 = V_{cc} - U_1$ between the resistors. It is independent from the total current and only depending on the ratio of the two resistors. It would not make a difference if we use $R_1 = \SI{10}{K\Omega}, R_2 = \SI{2,2}{K\Omega}$ or $R_1 = \SI{10}{\Omega}, R_2 = \SI{2,2}{\Omega}$, the result would be the same.


\begin{figure}[htb]
	\centering
	\begin{tikzpicture}
		\draw (0,0) node[ground] {}
			to[resistor, l=$R_2$, -*] ++ (0,2) coordinate (A)
			to[short, -o] ++ (1,0)
			node[auto, anchor=west] {$U_o$}
			(A) to[resistor, l=$R_1$] ++ (0,2)
			node[vcc] {$V_{cc}$}
		;
	\end{tikzpicture}
	\caption{A voltage divider made with two resistors}
	\label{fig:simple-voltage-divider}
\end{figure}


We can use this circuit to reduce a voltage but its usability as a voltage source is limited. This can be shown in \cref{fig:loaded-voltage-divider} by adding a load resistor $R_L$. Now the total resistance of the parallel resistors $R_2$ and $R_L$ calculates to
\begin{equation*}
	R_x = \frac{1}{\frac{1}{R_2} + \frac{1}{R_L}}
\end{equation*}
and the result of this is smaller than $R_2$ and $R_L$. The load changes the voltage divider and this changes the output voltage $U_o$.


\begin{figure}[htb]
	\centering
	\begin{tikzpicture}
		\draw (0,0) node[ground] {}
			to[resistor, l=$R_2$, -*] ++ (0,2) coordinate (A)
			to[short, -o] ++ (1,0) coordinate (B)
			node[auto, anchor=west] {$U_o$}
			(A) to[resistor, l=$R_1$] ++ (0,2)
			node[vcc] {$V_{cc}$}
			(B) to[resistor, l=$R_L$] ++ (0,-2)
			to[short, -*] ++ (-1,0)
		;
	\end{tikzpicture}
	\caption{A loaded voltage divider}
	\label{fig:loaded-voltage-divider}
\end{figure}

A voltage divider can only be used as a voltage source if there is no load attached to it. This means we cannot use it to drive an electric device but we will later see that there is a possibility to avoid this issue. Right now we only have discussed what happens when we have a fixed voltage $V_{cc}$ driving this circuit. With an alternating voltage $U = U_0 \cdot \sin(\omega \cdot t)$ we will get into trouble.

\subsection{The capacitive voltage divider}

In theory a resistor should only have a resistance but in reality every component has some resistance, capacitance and inductance. In many cases we can ignore the unwanted side-effects but when the circuit has to interact with high frequency signals these have to be taken into account. \todo[inline]{maybe show a picture of a high frequency signal which is distorted by a voltage divider}. At high frequencies the voltage divider can also be build with the use of capacitors instead of resistors.
